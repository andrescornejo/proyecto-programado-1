\documentclass{article}

%Formatting
\usepackage[utf8]{}
\usepackage[spanish]{babel}
\usepackage[backend=biber, style=authoryear-icomp]{biblatex}
\addbibresource{ref.bib}
\author{Andrés Cornejo, Ángel Villalobos}
\date{Noviembre 18, 2020}

\title{Machine learning y su factibilidad en los casos de uso modernos}

\begin{document}

\maketitle

% Abstract
\begin{abstract}
  TODO
\end{abstract}

\section{Introducción}
En la ciencia de la computación, uno de los auges más grandes de la última década es el de la inteligencia artificial. La rama de la computación que en los años ochenta era considerada intratable por su alta demanda de recursos computacionales, hoy es utilizada en todos los aspectos del diario vivir de los seres humanos. Algunos ejemplos de esto son GPS, servicios de streaming, artículos y vídeos presentados por redes sociales, filtros de datos, etc. Sin embargo, ¿Qué tan factible es el uso de algoritmos de Machine Learning para casos de uso modernos, desde un punto de vista de recursos computacionales? Resolver esta pregunta es el objetivo principal de esta investigación. Específicamente, se va a abordar un algoritmo de Machine Learning supervisado (Árbol de decisión), y otro de Machine Learning no supervisado (K-Means clustering). Otras preguntas planteadas a lo largo de esta investigación son: ¿Qué tan demandante es el algoritmo de árbol de decisión en un contexto moderno?, ¿Qué tan demandante es el algoritmo de clustering K-Means en un contexto moderno?, y finalmente, ¿Qué tan comparables son los algoritmos de Machine learning supervisado comparados a los no supervisados? (Específicamente árbol de decisión versus K-Means).

Para resolver estas preguntas, el método de resolución es el siguiente. Primero se programará cada algoritmo. Una vez programados, para cada uno, individualmente, se realizará una medición empírica.
La estructura de este artículo va a ser la siguiente,
\printbibliography

\end{document}
